\documentclass[
    english, % Klasei padavus parametrą 'english', darbas bus anglų kalba.
    % signatureplaces % prideda parašų vietas tituliniame puslapyje
]{VUMIFPSkursinis}
\usepackage{float}
\usepackage{wrapfig2}
\usepackage{hyperref}
\usepackage{algorithmicx}
\usepackage{algorithm}
\usepackage{algpseudocode}
\usepackage{amsfonts}
\usepackage{amsmath}
\usepackage{bm}
\usepackage{caption}
\usepackage{color}
\usepackage{graphicx}
\usepackage{listings}
\usepackage{subcaption}
\usepackage{biblatex}
\usepackage{minted}

% Titulinio aprašas
\university{Vilniaus universitetas}
\faculty{Matematikos ir informatikos fakultetas}
\department{Programų sistemų studijų programa}
\papertype{Kursinis darbas}
\title{Duomenų Struktūros ir Algoritmai Moderniame C++}
\titleineng{Modern C++ for Data Structures and Algorithms}
\author{Domas Kalinauskas}
\status{4 kurso I grupės studentas}
% \secondauthor{Vardonis Pavardonis}   % Pridėti antrą autorių
\supervisor{Viktoras Golubevas}
\reviewer{doc. dr. Vardauskas Pavardauskas}
\date{Vilnius – \the\year}

\bibliography{bibliografija}

\begin{document}
\maketitle

\tableofcontents

\sectionnonum{Įvadas/Intro?}

\subsection{Tikslas/Aim}

The aim of this <Kursinis darbas> is to analyze how modern C++ improves on
the usability of DSA's (data structures and algorithms) and their performance, in comparison to older versions
of the C++ standard, and how it stacks up with other languages.

\subsection{Uždaviniai/Objectives}

1. 
2.
3.

\subsection{Ivadas/Intro}

Collectively, the world is moving towards safer and safer languages. This is seen via widespread adoption of memory-safe languages,
such as Rust, or the overall popularity of 'managed', garbage-collected runtimes such as Java, .NET, Python etc.
(The safety in 'managed' languages comes from the fact that a vulnerability in the runtime has a much smaller surface area - requiring only a fix in the runtime, instead of the code that is built on top of it)

However, C++ has seen somewhat of a 'renessaince', with a similiar move to safer and more declarative APIs, along with a push towards 'functional' style concepts.

This <Kursinis darbas> aims to be a comprehensive comparison-analysis of recently introduced modern C++ features, and how they stack up against similar features in other languages.
We'll <galiu naudot?> take a look at the memory impact, performance impact, and implementation details of these features, along with where they might be ideally used.

Įvade apibūdinamas darbo tikslas, temos aktualumas ir siekiami rezultatai.
Darbo įvadas neturi būti dėstymo santrauka. Įvado apimtis 1–-2 puslapiai.

\section{Modern C++ Features for Algorithmic Problem Solving}

\subsection{Ranges}

A common pattern in C++ code is applying an operation over a selection of elements. The most classic style, which is still in-use today is the index based loop:
\inputminted[]{C++}{./source_code/ranges_loop_index.cpp}
There are a few notable downsides to the index based for loop, namely that it's error-prone, and can only be effectively used with random-indexable types (e.g. std::array, std::vector).
This means that if we want to iterate over a list, or another custom container, we'd have to change the for loop structure to be compatible.


With the introduction of C++11, we got access to the range-based for loop:
\inputminted[]{C++}{./source_code/ranges_loop_range.cpp}
When using range-based for loops, we no longer have to manually write the iteration code, meaning it doesn't matter if the type is a array, vector, list, or any other custom iterator.
Under the hood, these range-based for loops relies on the container having .begin()/.cbegin() and .end()/.cend() member functions.

% However, we're still constrained to using a for loop, which is less than ideal, since we can't access precisely X elements without adding a manual counter \& break statement.

\subsection{Concepts}
\subsection{Functional}
(Variants, Optional, Expected, Transform, t.t.)

Compare with equivalent features in Rust (mixed-paradigm language) \& Haskell (functional language)

\section{Modern C++ for High-Performance Computing}

\subsection{Compile-time calculations}
* Compare with other languages that can do that (Zig?)
* Mention best compile-time LUT library \:\^)
* Maybe something about compression at compile-time

\subsection{Execution policies}
* Reference TBB
* Can be compared to Rust libraries with similiar thing (ehhh forgot but iterator)

\subsection{Coroutines}
* reference coroutine look-up? (https://www.youtube.com/watch?v=j9tlJAqMV7U)
* compare to Rust async \& maybe js? async

\section{Medžiagos darbo tema dėstymo skyriai}
Medžiagos darbo tema dėstymo skyriuose pateikiamos nagrinėjamos temos detalės:
pradinė medžiaga, jos analizės ir apdorojimo metodai, sprendimų įgyvendinimas,
gautų rezultatų apibendrinimas. Šios dalies turinys labai priklauso nuo darbo
temos. Skyriai gali turėti poskyrius ir smulkesnes sudėtines dalis, kaip
punktus ir papunkčius.

Medžiaga turi būti dėstoma aiškiai, pateikiant argumentus. Tekste dėstomas
trečiuoju asmeniu, t.y. rašoma ne \enquote{aš manau}, bet „autorius mano“, „autoriaus
nuomone“. Reikėtų vengti informacijos nesuteikiančių frazių, pvz., „...kaip jau
buvo minėta...“, \enquote{...kaip visiems žinoma...} ir pan., vengti grožinės
literatūros ar publicistinio stiliaus, gausių metaforų ar panašių meninės
išraiškos priemonių.

Skyriai gali turėti poskyrius ir smulkesnes sudėtines dalis, kaip punktus ir
papunkčius.

\subsection{Poskyris}
Citavimo pavyzdžiai: cituojamas vienas šaltinis \cite{PvzStraipsnLt}; cituojami
keli šaltiniai \cite{PvzStraipsnEn, PvzStraipsnLta, PvzKonfLt, PvzKonfEn, PvzKnygLt, PvzKnygEn,
PvzElPubLt, PvzElPubEn, PvzBakLt, PvzMagistrLt, PvzPhdEn}.

Anglų kalbos terminų pateikimo pavyzdžiai: priklausomybių injekcija (\anglnb{dependency injection},
dažnai trumpinama kaip \textit{DI}), saitų redaktorius \angl{linker}.

Išnašų\footnote{Pirma išnaša.} pavyzdžiai\footnote{Antra išnaša.}.

\subsection{Faktorialo algoritmas}

\ref{alg:factorial} algoritmas parodo, kaip suskaičiuoti skaičiaus faktorialą.

\begin{algorithm}
\caption{Skaičiaus faktorialas}
\begin{algorithmic}[1] % [1] padaro, kad eilutės būtų sunumeruotos
\State $N\gets$ skaičius, kurio faktorialą skaičiuojame
\State $F\gets 1$
\For{$i := 2$ to $N$}
    \State $F\gets F \cdot i$
\EndFor
\end{algorithmic}
\label{alg:factorial}
\end{algorithm}

\subsubsection{Punktas}
\subsubsubsection{Papunktis}
\subsubsection{Punktas}
\section{Skyrius}
\subsection{Poskyris}
\subsection{Poskyris}

\sectionnonum{Rezultatai ir išvados}
Rezultatų ir išvadų dalyje turi būti aiškiai išdėstomi pagrindiniai darbo
rezultatai (kažkas išanalizuota, kažkas sukurta, kažkas įdiegta) ir pateikiamos
išvados (daromi nagrinėtų problemų sprendimo metodų palyginimai, teikiamos
rekomendacijos, akcentuojamos naujovės).

\sectionnonum{Išvados}
\begin{enumerate}[labelindent=0pt]
    \item Išvadų skyriuje daromi nagrinėtų problemų sprendimo metodų palyginimai, siūlomos
rekomendacijos, akcentuojamos naujovės.
    \item Išvados pateikiamos sunumeruoto (gali būti hierarchinis) sąrašo pavidalu.
    \item Darbo išvados turi atitikti darbo tikslą.
\end{enumerate}

\printbibliography[heading=bibintoc]  % Šaltinių sąraše nurodoma panaudota
% literatūra, kitokie šaltiniai. Abėcėlės tvarka išdėstomi darbe panaudotų
% (cituotų, perfrazuotų ar bent paminėtų) mokslo leidinių, kitokių publikacijų
% bibliografiniai aprašai. Šaltinių sąrašas spausdinamas iš naujo puslapio.
% Aprašai pateikiami netransliteruoti. Šaltinių sąraše negali būti tokių
% šaltinių, kurie nebuvo paminėti tekste (LaTeX tai sutvarko automatiškai).

% \sectionnonum{Sąvokų apibrėžimai}
\sectionnonum{Santrumpos}
Sąvokų apibrėžimai ir santrumpų sąrašas sudaromas tada, kai darbo tekste
vartojami specialūs paaiškinimo reikalaujantys terminai ir rečiau sutinkamos
santrumpos.

% Priedai
% Prieduose gali būti pateikiama pagalbinė, ypač darbo autoriaus savarankiškai
% parengta, medžiaga. Savarankiški priedai gali būti pateikiami ir
% kompaktiniame diske. Priedai taip pat numeruojami ir vadinami. Darbo tekstas
% su priedais susiejamas nuorodomis.
\appendix{Neuroninio tinklo struktūra}

\begin{figure}[H]
    \centering
    \includegraphics[scale=0.5]{img/MLP}
    \caption{Paveikslėlio pavyzdys}
    \label{img:mlp}
\end{figure}


\appendix{Eksperimentinio palyginimo rezultatai}

% tablesgenerator.com - converts calculators (e.g. excel) tables to LaTeX
\begin{table}[H]\footnotesize
  \centering
  \caption{Lentelės pavyzdys}
  {\begin{tabular}{|l|c|c|} \hline
    Algoritmas & $\bar{x}$ & $\sigma^{2}$ \\
    \hline
    Algoritmas A  & 1.6335    & 0.5584       \\
    Algoritmas B  & 1.7395    & 0.5647       \\
    \hline
  \end{tabular}}
  \label{tab:table example}
\end{table}

\end{document}
